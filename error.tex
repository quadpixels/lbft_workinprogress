%%
%% This is file `error.tex',
%% generated with the docstrip utility.
%%
%% The original source files were:
%%
%% acmconf.dtx  (with options: `error,body')
%% 
%% IMPORTANT NOTICE:
%% 
%% For the copyright see the source file.
%% 
%% Any modified versions of this file must be renamed
%% with new filenames distinct from error.tex.
%% 
%% For distribution of the original source see the terms
%% for copying and modification in the file acmconf.dtx.
%% 
%% This generated file may be distributed as long as the
%% original source files, as listed above, are part of the
%% same distribution. (The sources need not necessarily be
%% in the same archive or directory.)
%%
%% to test the checksum, uncomment    \OnlyDescription
%% in the driver
%%\fi
%%
%% \CharacterTable
%%  {Upper-case    \A\B\C\D\E\F\G\H\I\J\K\L\M\N\O\P\Q\R\S\T\U\V\W\X\Y\Z
%%   Lower-case    \a\b\c\d\e\f\g\h\i\j\k\l\m\n\o\p\q\r\s\t\u\v\w\x\y\z
%%   Digits        \0\1\2\3\4\5\6\7\8\9
%%   Exclamation   \!     Double quote  \"     Hash (number) \#
%%   Dollar        \$     Percent       \%     Ampersand     \&
%%   Acute accent  \'     Left paren    \(     Right paren   \)
%%   Asterisk      \*     Plus          \+     Comma         \,
%%   Minus         \-     Point         \.     Solidus       \/
%%   Colon         \:     Semicolon     \;     Less than     \<
%%   Equals        \=     Greater than  \>     Question mark \?
%%   Commercial at \@     Left bracket  \[     Backslash     \\
%%   Right bracket \]     Circumflex    \^     Underscore    \_
%%   Grave accent  \`     Left brace    \{     Vertical bar  \|
%%   Right brace   \}     Tilde         \~}


\documentclass[print]{acmconf}

%% LaTeXing this file should produce some errors.

\pagestyle{headings}

\IfFileExists{graphicx.sty}{\usepackage{graphicx}}{}

\def\XX{More text should follow, but keep in mind that a limit of 6
  pages has been set, including figures and references.  More text
  should follow, but keep in mind that a limit of 6 pages has been
  set, including figures and references.  More text should follow, but
  keep in mind that a limit of 6 pages has been set, including figures
  and references.  More text should follow, but keep in mind that a
  limit of 6 pages has been set, including figures and references.
  \par
}

\begin{document}
\date{31. December 1999}
\title{A New Intuitionistic Proof of Usability\\
       of the Recommended Style File for the ACM Conference Papers}
\author{\Author{J\"urgen Vollmer\thanks{Happy \LaTeX{}ing}}\\
         \Address{Karlsruhe}\\
         \Email{Juergen.Vollmer@acm.org}\\
         \and
         \Author{Mickey Mouse}\\
         \Address{Enthausen University}\\
         \Email{Mickey.Mouse@entenhausen.org}
       }
\maketitle

\begin{abstract}
  This document demonstrates how to use the \LaTeX2e \verb|acmconf|
  class by exhibiting itself as an example.  You are expected to be
  familiar with~\cite{Lam94}.  The best way to use this file is to use
  it as a template, i.e., replace the prose in it by your
  own\footnote{And may use footnotes.}.
\end{abstract}

\begin{keywords}
\LaTeX2e-class, ACM proceedings
\end{keywords}

\section{Introduction}
To understand this file read the \emph{source} and not the typeset
version.  If you are reading this in the typeset version you might as
well stop --- it is not supposed to make sense.

\section{The Story Begins\ldots}
A real article is supposed to have some deep results and good
explanations.  That, however, is your job and not mine so you should
replace this text with something more appropriate\footnote{Another a
  footnote}..

\section{Some often used \LaTeX\ commands}

\subsection{\texttt{emph}, etc.}
Text may be set as \emph{emph}.\\
Text may be set as \texttt{texttt}.\\
Text may be set as \underline{unterline}.\\
Text may be set as \textbf{textbf}.\\
Text may be set as \textrm{textrm}.\\
Text may be set as {\tiny tiny}.\\
Text may be set as {\scriptsize scriptsize}.\\
Text may be set as {\footnotesize footnotesize}.\\
Text may be set as {\normalfont normalsize}.\\
Text may be set as {\large large}.\\
Text may be set as {\Large Large}.\\
Text may be set as {\LARGE LARGE}.\\
Text may be set as {\huge huge}.\\
Text may be set as {\Huge Huge}.\\
Text may have$^{\textrm{super}}$ and$_{\textrm{sub}}$scripts.

\subsection{\texttt{itemize}}
\begin{itemize}
\item More text should follow, but keep in mind that a limit of 6
     pages has been set, including figures and references.  More text
     should follow, but keep in mind that a limit of 6 pages has been
     set, including figures and references.
\item More text should follow, but keep in mind that a limit of 6
     pages has been set, including figures and references.  More text
     should follow, but keep in mind that a limit of 6 pages has been
     set, including figures and references.
\end{itemize}

\subsection{\texttt{enumerate}}
\begin{enumerate}
\item More text should follow, but keep in mind that a limit of 6
     pages has been set, including figures and references.  More text
     should follow, but keep in mind that a limit of 6 pages has been
     set, including figures and references.
\item More text should follow, but keep in mind that a limit of 6
     pages has been set, including figures and references.  More text
     should follow, but keep in mind that a limit of 6 pages has been
     set, including figures and references.
\end{enumerate}

\subsection{\texttt{description}}
\begin{description}
\item[Foo] More text should follow, but keep in mind that a limit of 6
     pages has been set, including figures and references.  More text
     should follow, but keep in mind that a limit of 6 pages has been
     set, including figures and references.
\item[Bar] More text should follow, but keep in mind that a limit of 6
     pages has been set, including figures and references.  More text
     should follow, but keep in mind that a limit of 6 pages has been
     set, including figures and references.
\end{description}

\subsection{\texttt{center} and \texttt{tabular}}
\begin{center}
\begin{tabular}{|l|c|r|}\hline
left     & center   & right    \\\hline\hline
AAAAAAAA & BBBBBBBB & CCCCCCCC \\
AAAAAAAA & BBBBBBBB & CCCCCCCC \\\cline{3-3}
AAAAAAAA & BBBBBBBB & CCCCCCCC \\\cline{2-2}
AAAAAAAA & BBBBBBBB & CCCCCCCC \\\cline{1-2}
AAAAAAAA & BBBBBBBB & CCCCCCCC \\\hline
AAAAAAAA & BBBBBBBB & CCCCCCCC \\\hline
1          & \multicolumn{2}{|c|}{2} \\\hline
\end{tabular}
\end{center}

\subsection{\texttt{figure} and Postscript pictures}
Have a look to to figure~\ref{fig-1} and~\ref{fig-2}.

\begin{figure}
\hrule
Nice Postscript, isn't it?
\begin{center}
\IfFileExists{graphicx.sty}{
  \includegraphics{body.eps}
}{
  Sorry, package \texttt{graphicx} not present.
}
\end{center}

Same, a little bit smaller:
\begin{center}
\IfFileExists{graphicx.sty}{
  \includegraphics[scale=.5]{body.eps}
  }{
  Sorry, package \texttt{graphicx} not present.
}
\end{center}
\caption{\label{fig-1}This is a nice floating figure}
\hrule
\end{figure}

\begin{figure*}
\hrule
This figure uses both columns, using \texttt{figure*}
\begin{center}
\IfFileExists{graphicx.sty}{
  \includegraphics[scale=.5]{body.eps}
  \hspace{1cm}
  \includegraphics[scale=.5]{body.eps}
}{
  Sorry, package \texttt{graphicx} not present.
}
\end{center}
\caption{\label{fig-2}This is a nice floating figure}
\hrule
\end{figure*}

\section{The Story Continues 1}

This is a \verb+\section+.

\XX\XX

\subsection{The Story Continues 2}

This is a \verb+\subsection+.

\XX\XX

\subsubsection{The Story Continues 3}

This is a \verb+\subsubsection+.

\XX\XX

\subsubsubsection{The Story Continues 4}

This is a \verb+\subsubsubsection+.

\XX\XX

\subsubsubsubsection{The Story Continues 5}

This is a \verb+\subsubsubsubsection+.

\XX\XX

\paragraph{The Story Continues 6}

This is a \verb+\paragraph+.
\XX\XX

\subparagraph{The Story Continues 7}
This is a \verb+\subparagraph+.
\XX\XX\XX

\section{Conclusion}
The end, at last!  In this example there really are no results or
points to summarize but I trust your article has more food for though
and thus will need a conclusion.

\appendix
\section{Appendices}
If you have any, appendices might go here.  Note that appendices
should not be used to circumvent the word count limit.

This is "doing it by hand" --- you might be better off using BibTeX.

\begin{thebibliography}{X}
\bibitem[1]{Lam94} Leslie Lamport: {\em \LaTeX, A Document
    Preparation System,} Addison Wesley~1994.
\end{thebibliography}
\IfPrepare{
  \tableofcontents
  \listoffigures
  \listoftables
}{}
\tableofcontents
\listoffigures
\listoftables
\end{document}
\endinput
%%
%% End of file `error.tex'.
