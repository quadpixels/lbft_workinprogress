\documentclass[prodmode,acmtecs]{acmconf} % Aptara syntax
\usepackage{url}
\usepackage{graphicx}
\usepackage{amsmath}
\usepackage{color}
\usepackage{multirow}
\usepackage[ruled]{algorithm2e}

\newcommand{\sui}[1]{%
  \textcolor{green}{SC - #1}
}

\newcommand{\marc}[1]{%
  \textcolor{red}{[MC: #1]}
}

\newcommand{\greg}[1]{%
  \textcolor{blue}{GB: #1}
}

\ConferenceShortName{ICS}
\ConferenceName{International Conference on Supercomputing}

\title{Library-based fault tolerance for scientific applications}
\author{Sui Chen
%\affil{Louisiana State University}
Greg Bronevetsky, Marc Casas-Guix
%\affil{Lawrence Livermore National Laboratory}
}

\begin{document}

\maketitle

\section{Introduction}
\label{sec:intro}

The increasing size and complexity of (High-Performance Computing) HPC systems is making them increasingly vulnerable to soft faults.
Soft faults are transient corruptions of circuit state caused by physical phenomena such as strikes by neutrons or alpha particles~\cite{baumann:2005, asciQSER:2005} or thermal electrical noise~\cite{therm_noise:2007}.
They can affect the state of processor latches and registers and may cause the application to crash or worse, to silently return incorrect results~\cite{mpi_ser:reed:2004}.
As the electronic feature sizes shrink technology scaling will make soft errors a larger problem~\cite{err_scaling:2012} due to the fact that each circuit element will hold less charge and can thus be disrupted by easily.
These phenomena make it imperative to develop techniques to make HPC systems resilient to soft faults.

Resilience is a problem that must be addressed at all levels.
While materials science and circuit design techniques can be used to improve resilience, they come at a cost in reduced power efficiency and performance that an become prohibitive if processors need to be sufficiently reliable to build a large HPC system.
Techniques such as error correcting codes have been very effective at making memories and caches resilient to soft faults~\cite{mem_errors:2010}.
However, they are more expensive for protecting core-internal state such as latches and are significantly less effective for checking the correctness of computations.
Traditional approaches like cross-core or node replication~\cite{rmpi:2011, dyn_cmp_repl:2007} are expensive in their use of computing resources and power and their long error detection latencies require processor state to be frequently checkpointed.
Processor designs that incorporate instruction replication~\cite{repl_smt:2000} offer fine-grained error detection and rollback but require more power as well as novel hardware features that are not likely to be included in the commodity processors must be used in HPC systems to limit their cost.

The limitations of automated resilience techniques has motivated the need for algorithmic techniques that can detect and tolerate errors by taking advantage of application-specific properties and invariants~\cite{amg_abft:2012, robustification:2010, 1bft:1984}.
Although these techniques can be very efficient, they also require manual effort to incorporate into applications and may require deep insight into an application's invariants to deploy effectively.
Fortunately, because real-world applications spend most of their execution is spent in a few application modules, most soft faults will occur in those modules rather than uniformly throughout the application.
In scientific applications these modules are usually various types of libraries, including numeric solvers and domain-specific scientific packages.
This indicates that if by focusing algorithmic resilience efforts on the key library routines, while leaving other modules unprotected or protected via more expensive mechanisms, it is possible to fully protect an application at a low cost.

This paper evaluates this approach in the context of three different types of applications: the MPI implementation of using the Alternating Direction Method of Multipliers (ADDR) to solver Lasso problems~\cite{lasso:2011}, the Hattrick gravitational simulation~\cite{hattrick:2012} and the Digital Room Correction (DRC) acoustic correction application~\cite{drc:2012}.
We show how these three applications can be comprehensively protected from soft faults by adding three different resilience mechanisms to the routines where they spend the most time: algorithmic error checks, replication of critical data structures and checkpoint-restart of individual routines.
We demonstrate via fault injection experiments that although none of these techniques can individually protect applications from soft faults, their combination is highly effective.
Further, we demonstrate how these techniques can be parameterized to offer the right level of protection at any given fault rate or input type, and further can be tuned to bound the probability that the application will produce erroneous results or executes for longer than some limit.

\section{Soft Faults}
\label{sec:soft_faults}

Causes of soft faults and how they manifest themselves in applications.

Discuss prior work on replication and algorithmic techniques, emphasize that in this paper we're focusing on software resilience since hardware resilience is extremely difficult to deploy in real processors since the costs have to be carried by non-HPC markets that care much less about it.

Can show plots of how routine results are affected by injected errors. Should show this for all the routines we'll be considering in this paper.

Describe fault injection framework and how it approximates real faults.

\section{Target Applications}
\label{sec:apps}

We evaluate our library-based resilience approach by applying it to the following three applications.
As they make intensive use library routines, they are ideal testbeds for our proposed fault-tolerance methods.

\greg{This section needs to include more detailed descriptions of the algorithms, the routines where they spend the bulk of their time and what fraction of time is spent there across the range of inputs. We also need to describe the types of inputs we'll be giving them in our experiments.}
\subsection{Lasso}
\label{sec:apps:lasso}
A parallel shrinkage and selection method for linear regression. It computes the equation $Ax=b$, where matrix $A$ and vector $b$ are known. \greg{is it a shrinkage and selection method or an actual solver?}. Utilizes GSL (the GNU Scientific Library)'s linear algebra routines.

\subsection{DRC: Digital Room Correction}
\label{sec:apps:drc}

A program used to generate correction filters for acoustic compensation of HiFi and audio system in general. It generates FIR correction filters, which can be used with a real time or offline convolver to provide correction. Utilizes the GSL FFT routine.

\subsection{Hattrick}
\label{sec:apps:hattrick}
An accurate N-Body integrator meant for high control over small systems, specifically meant for perturbing bodies. Utilizes a differential equation solver routine.

\section{Resilience Techniques}
\label{sec:res_tech}

\subsection{Error Detection}
\label{sec:res_tech:err_det}

\subsubsection{Algorithmic Detection}
\label{sec:res_tech:err_det:algo}

\greg{This section needs performance results. How much slower are the resilient routines and how much more resilient do they become across a range of detection thresholds.}
\sui{Performance experiments inserted here: running algorithmic checkers across different input sizes and fault rates: 62x62 to 500x500 matrixes for MM, RK, CD, MV checkers; FFT sizes of $2^8$ to $2^{24}$ for the FFT checker; Error bars should apply here, especially for the Cholesky Decomposition checker}

\paragraph{Matrix-matrix multiplication (MMM)}:
Matrix-matrix multiplications are checked using a matrix vector multiplication, making use of this identity: $(A \cdot B) \cdot x = A \cdot (B \cdot x)$. Here x is a generated error-checking vector and $A$ and $B$ are inputs to the matrix-matrix multiplication. Time complexities of the original routine and the checker are $O(n^3)$ and $O(n^2)$ respectively.


\paragraph{Matrix-vector multiplication (MVM)}:
For matrix-vector multiplication $Ax=b$ , note that the sum of the elements in the result vector $b$ is equal to the dot product of the vector composed of column sums of $A$ and $x$. Time complexity of original routine is $O(n^2)$ multiplications and that of the checker is $O(n^2)$ additions.

\paragraph{Symmetric Rank-K update (SYRK)}:
Symmetric Rank-K update is a special case of matrix-matrix multiplication where only the upper/lower half of the output matrix is updated. By decomposing the upper/lower half of the output matrix as a series of sub-matrices (like the sub-cells in a quad tree), the checker takes $O(n^2 \cdot \log{n})$ time, while the original routine woudl take $O(n^3)$.

\paragraph{Cholesky Decomposition}: \greg{Need to define this decomposition}
By multiplying back the upper and lower halves, the checker takes as much time as a regular matrix-matrix multiplication would do. The original routine is an iterative one, but it generally runs longer than the checker. \greg{How much longer? Need to quantify.} \sui{Can we quantify it using the performance data?}

\paragraph{Fast Fourier Transform (FFT)}:
By using Parseval's Theorem, we can check the result of an FFT of width $n$ in $O(n)$ time. Time complexity of the original routine is $O(\log{n})$.
\greg{What is this theorem, how is it used?}

\paragraph{Runge-Kutta PDE Solver (RK)}:
\greg{Describe the multiple step size checker in Hattrick}.

We evaluate these algorithmic error checkers by measuring their costs and effectiveness when running at different error rates on different types of inputs.

Experiment 1: Measure performance degradation of using error detectors as input sizes increase. In case of RK if we can run a version that doesn't do the multiple step sizes, do so but if this is difficult then don't worry about it.

Experiment 2: Measure the error in the outputs of the algorithms with and without the algorithmic detectors.
First, show a probability distribution histogram for all the codes at one error rate, input size and algorithmic detection threshold.
Then show a summary of the entire space using the same heatmaps that we're using for final application results.
Put the input size and error rate on the y-axis and the detection threshold on the x-axis.
The color in each box should be the fraction of entries in the output that have an error of a given magnitude.
Show a separate plot for a few different magnitudes.

Experiment 3: Measure the fraction fraction of routine runs that are interrupted by the detection of an error.
Display this information via the same heatmaps as above.

For each experiment we need to include a discussion of what it means for how these checkers behave within the full-application runs.

%Those checkers can detect errors greater than a certain numeric threshold in those computations (determined by the user-defined error checker threshold). Once an error beyond the threshold has occurred, a re-calculation would be initiated.
%\greg{You need to be more specific. Each routine produces a result, your checker produces a result. How are these results compared and how is the threshold used to determine if there is an error or not.}

%In the following sections we would show that the error tolerance could not be arbitrarily small in order to be helpful in providing fault-tolerance when we take into account the checker itself may be unreliable since it's also vulnerable to soft errors.
%\greg{How is this shown? We certainly do show tighter detection tolerances cause longer executions, which cause more vulnerability. However, this is not the same as what you've just described}

%Although checkers are faster than the routines they protect, the unreliability of those checkers and the ensuing ``false alarms'' might trigger redundant re-calculations that are unnecessary, causing the application to run for much longer. This is specially true with an overly small error detection threshold. On the other hand, an overly loose detection threshold may fail to detect data corruption in output files. The results would be discussed in the results section.

\subsubsection{Memory Fault Detection}
\label{sec:res_tech:err_det:mem}

Modern operating systems and hardware efficiently detect accesses to memory regions that are not allocated to the application and make it easy for the application to detect such errors by catching Segmentation Fault and Bus Error signals.
We have used this mechanism to protect applications from a subset of memory errors.


Experiment 1: Show the fraction of runs of each routine that is aborted as a result of a memory error

Errors that do not cause the application to access an invalid memory location are detected by replicating key pointers and loop iteration variables.
The Hattrick application uses a high-accuracy ODE solver that is capable of estimating errors and detecting and reducing errors to a certain level adaptively.
Therefore we would only focus on the application's vulnerability to pointer and index corruptions.
Replicating critical variables (pointers and indices) and leveraging Byzantine fault tolerance and correcting those pointers and indices during run-time would raise the applications' resistance to soft errors.
\greg{Need to explain why we only do this for Hattrick.}

To determine a replication strategy we have to determine which variables to replicate and how many replicas to use for replicating them. The following two subsections would be focused on these problems.
\greg{The report spends a lot of space talking about this but in the end the conclusion is that we just a pick a few options and evaluate them experimentally. How much space should we devote to this discussion.}

Experiment 2: Measure the effectiveness and performance costs of replication. Need to explain why we don't use it for Lasso and DRC.

\subsection{Recovery via Checkpoint-Restart}
\label{sec:res_tech:cr}

When errors are detected we employ a recovery method based on checkpoint-restart.
At the start of each routine the state of all inputs is recorded and an error correcting code is used to ensure that the contents of this backup copy are not corrupted.
Further, we use the \texttt{sigsetjmp} routine to record the application's execution state at the routine's start.
If an error is detected during the routine \texttt{siglongjmp} is used to return the application's execution back to the start of the routine (this function unrolls function calls as needed and reloads the process counter and register state).
The recovery procedure then resets the routine's inputs from their backed-up copies, overwriting any changes or corruptions that may have occurred during the routine's execution, and restarts its execution.
%A combination of \texttt{siglongjmp} and \texttt{sigsetjmp} are used to recover from segmentation faults. The overhead of this API call is having to copy the processor state into memory, therefore installing them between computation-intensive method calls adds to negligible performance degradation.

The backup copy of routine inputs is protected via a block-checksum-based data correction mechanism.
By treating the input array as a matrix and comparing row sums and column sums of the submatrices of that matrix, we are able to detect the existence of errors and fix a fraction of the errors.
The protection mechanism has the following parameters:
\begin{itemize}
\item{\texttt{N}, the size of the error correcting sum block.}
\item{\texttt{ECC\_ECC}, whether or not the error correcting code itself should be corrected.}
\end{itemize}
This mechanism is able to correct up to $1/{n^2}$ of the elements in the input array.
\greg{Which code is being used here? You just say that it is based on block-checksums}.

Experiment 1: Measure the cost of setjmp and longjmp in our routines as the input sizes vary
Experiment 2: Measure the cost of input backup and input recovery in our routines as the input sizes vary

\section{Evaluation of Resilience Methodology}
\label{sec:eval}

\greg{This is the section where we put all the pieces together and present the experimental results on full applications. We should describe our main use-case here: user wants to achieve a given accuracy with a given probability and we configure the algorithm to provide this at the lowest cost. QUESTION: how do we pick the parameters of each resilience technique? Which metric is being optimized?} \sui{Using the choices of those parameters derived from the experiments we have already performed (treat it as a ``training set'') and use those choices on some new experiment configurations (treat as ``test set''), and see what is the probability that those choices could satisfy the users' experiments.}

\subsection{Performance Impact}
\label{sec:eval:perf}

Show the Markov model that describes our rollback scheme.

Experiment 1: Measure the slowdown of each application at the optimal configuration
Experiment 2: Measure the rate of aborts due to algorithmic errors, segfaults and redundant copy mismatches (separate the causes)

\subsection{Result Accuracy}
\label{sec:eval:acc}

Experiment: Measure the accuracy of the results produced in completed runs, including error bars on the variability of the errors

\subsection{Overall Evaluation}
\label{sec:eval:overall}

Experiment: Final graph that shows slowdown for each desired probability of perfect results.

\section{Summary}
\label{sec:summary}

\end{document}
